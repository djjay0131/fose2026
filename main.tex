\documentclass[10pt]{article}

% -------------------------
% Packages
% -------------------------
\usepackage[utf8]{inputenc}
\usepackage[T1]{fontenc}
\usepackage{times}
\usepackage{microtype}
\usepackage{graphicx}
\usepackage{amsmath}
\usepackage{amssymb}
\usepackage{booktabs}
\usepackage{hyperref}
\usepackage{xcolor}
\usepackage{enumitem}
\usepackage{caption}
\usepackage{subcaption}

\hypersetup{
  colorlinks=true,
  linkcolor=blue,
  citecolor=blue,
  urlcolor=blue
}

% -------------------------
% Title & Author (placeholder)
% -------------------------
\title{
From Papers to Progress:\\
Rethinking Knowledge Accumulation in Software Engineering
}

\author{
Anonymous for Review
}

\date{}

\begin{document}
\maketitle

% -------------------------
% Abstract
% -------------------------
\begin{abstract}
Software engineering research has experienced sustained growth in both output and participation over the past decades. Yet concerns persist about the field’s ability to accumulate, integrate, and reuse knowledge in ways that support long-term progress. To better understand how the community itself perceives these challenges, we analyze responses from the ICSE 2026 Future of Software Engineering pre-survey, which captures perspectives from a globally distributed and highly experienced set of researchers. Our analysis reveals a tension between increasing research productivity and the limited mechanisms available for synthesizing results, tracking evolving claims, and supporting cumulative understanding over time. Building on these observations, we argue that future progress in software engineering will require rethinking the dominant research artifacts and infrastructures that shape how knowledge is produced, evaluated, and reused. We outline a set of principles for more cumulative, interpretable, and provenance-aware research artifacts, and discuss their implications for the future of software engineering research and community practice.
\end{abstract}

% -------------------------
% 1. Introduction
% -------------------------
\section{Introduction}
\label{sec:introduction}

% Motivation:
% - SE research output is growing
% - Participation is high
% - Yet knowledge fragmentation persists

% Contributions:
% - Community-informed perspective using FOSE pre-survey
% - Identification of structural barriers to cumulative knowledge
% - Forward-looking principles for research artifacts

\begin{itemize}[leftmargin=*]
  \item Context: growth of software engineering research output
  \item Persistent challenges in knowledge integration and reuse
  \item Why cumulative progress matters for the future of SE
  \item Contributions of this paper
\end{itemize}

% -------------------------
% 2. Background and Related Perspectives
% -------------------------
\section{Background and Related Perspectives}
\label{sec:background}

% Purpose:
% - Situate the paper within FOSE tradition
% - Avoid a full related work survey
% - Highlight recurring concerns about accumulation, replication, synthesis

\subsection{The Role of FOSE in Shaping the Field}
% FOSE as a venue for reflection, critique, and agenda-setting

\subsection{Knowledge Accumulation in Software Engineering}
% Prior discussions on:
% - Replication
% - Evidence synthesis
% - Research fragmentation
% - Artifact evaluation

% -------------------------
% 3. Community Signals from the ICSE 2026 FOSE Pre-Survey
% -------------------------
\section{Community Signals from the ICSE 2026 FOSE Pre-Survey}
\label{sec:survey}

% This section should stay high-level and descriptive

\subsection{Survey Context and Respondent Characteristics}
% Who responded
% Seniority
% Geographic distribution
% Research activity levels

\subsection{High-Level Observations}
% Key non-controversial signals:
% - Experience is not lacking
% - Participation is not lacking
% - Output volume is high

% Optional: include 1--2 summary figures or tables

% -------------------------
% 4. Where Knowledge Accumulation Breaks Down
% -------------------------
\section{Where Knowledge Accumulation Breaks Down}
\label{sec:breakdown}

% This is the intellectual core of the paper

\subsection{Papers as Isolated Knowledge Units}
% Claims, evidence, and context are weakly linked

\subsection{Loss of Context and Provenance Over Time}
% Difficulty understanding why results hold
% Difficulty tracing assumptions and evolution

\subsection{High Rediscovery and Reinterpretation Costs}
% Repeated problem formulation
% Limited reuse of prior insights

\subsection{Structural Reinforcement Through Incentives}
% Venues, review criteria, and career incentives

% -------------------------
% 5. Rethinking Research Artifacts for Cumulative Progress
% -------------------------
\section{Rethinking Research Artifacts for Cumulative Progress}
\label{sec:artifacts}

% Focus on principles, not systems

\subsection{From Documents to Structured Research Objects}
% Claims
% Evidence
% Context

\subsection{Interpretability and Inspectability}
% Supporting understanding by humans and machines

\subsection{Provenance and Evolution of Knowledge}
% Tracking how ideas change over time

\subsection{Long-Lived Knowledge Substrates}
% Beyond individual projects or papers

% -------------------------
% 6. Implications for the Future of Software Engineering
% -------------------------
\section{Implications for the Future of Software Engineering}
\label{sec:implications}

\subsection{Implications for Research Practice}
% How researchers design and report studies

\subsection{Implications for Publication and Review}
% Evaluation criteria
% Artifact expectations

\subsection{Implications for Community Infrastructure}
% Shared platforms
% FOSE as an experimentation space

% -------------------------
% 7. Limitations and Open Questions
% -------------------------
\section{Limitations and Open Questions}
\label{sec:limitations}

% Important for FOSE credibility
% - Survey scope
% - Interpretive nature of arguments
% - Open research challenges

% -------------------------
% 8. Conclusion
% -------------------------
\section{Conclusion}
\label{sec:conclusion}

% Reiterate:
% - Productivity is not the problem
% - Accumulation is the bottleneck
% - The future of SE depends on structural change

% -------------------------
% References
% -------------------------
\bibliographystyle{plain}
\bibliography{references}

\end{document}