\documentclass[sigconf,anonymous,review]{acmart}

\setcopyright{none}

%%
%% \BibTeX command to typeset BibTeX logo in the docs
\AtBeginDocument{%
  \providecommand\BibTeX{{%
    Bib\TeX}}}

%% Rights management information.  This information is sent to you
%% when you complete the rights form.  These commands have SAMPLE
%% values in them; it is your responsibility as an author to replace
%% the commands and values with those provided to you when you
%% complete the rights form.
% \setcopyright{acmcopyright}
% \copyrightyear{2026}
% \acmYear{2026}
% \acmDOI{XXXXXXX.XXXXXXX}

%%
%% These commands are for a PROCEEDINGS abstract or paper.
% \acmConference[ICSE '26]{48th International Conference on Software
%   Engineering}{April 12--18, 2026}{Rio de Janeiro, Brazil}
%%
%%  Uncomment \acmBooktitle if the title of the proceedings is different
%%  from ``Proceedings of ...''!
%%
% \acmBooktitle{Proceedings of the 48th International Conference on Software Engineering (ICSE '26)}
% \acmISBN{978-1-4503-XXXX-X/26/04}

%%
%% Submission id.
%% Use this when submitting an article to a sponsored event. You'll
%% receive a unique submission id from the organizers
%% of the event, and this id should be used as the parameter to this command.
%%\acmSubmissionID{123-A56-BU3}

%%
%% For managing citations, it is recommended to use bibliography
%% files in BibTeX format.
%%
%% You can then either use BibTeX with the ACM-Reference-Format style,
%% or BibLaTeX with the acmnumeric or acmauthoryear sytles, that include
%% support for advanced citation of software artefact from the
%% biblatex-software package, also separately available on CTAN.
%%
%% Look at the sample-*-biblatex.tex files for templates showcasing
%% the biblatex styles.
%%

%%
%% The majority of ACM publications use numbered citations and
%% references.  The command \citestyle{authoryear} switches to the
%% "author year" style.
%%
%% If you are preparing content for an event
%% sponsored by ACM SIGGRAPH, you must use the "author year" style of
%% citations and references.
%% Uncommenting
%% the next command will enable that style.
%%\citestyle{acmauthoryear}


%%
%% end of the preamble, start of the body of the document source.
\begin{document}

%%
%% The "title" command has an optional parameter,
%% allowing the author to define a "short title" to be used in page headers.
\title{From Papers to Progress: Rethinking Knowledge Accumulation in Software Engineering}

%%
%% The "author" command and its associated commands are used to define
%% the authors and their affiliations.
%% Of note is the shared affiliation of the first two authors, and the
%% "authornote" and "authornotemark" commands
%% used to denote shared contribution to the research.

%% Uncomment for anonymous review
% \author{Anonymous for Review}
% \renewcommand{\shortauthors}{Anonymous}

%% Author information (comment out for anonymous submission)
\author{Jason Cusati}
\affiliation{%
  \institution{Virginia Tech}
  \city{Blacksburg}
  \state{VA}
  \country{USA}
}
\email{djjay@vt.edu}

\author{Chris Brown}
\affiliation{%
  \institution{Virginia Tech}
  \city{Blacksburg}
  \state{VA}
  \country{USA}
}
\email{dcbrown@vt.edu}

%%
%% By default, the full list of authors will be used in the page
%% headers. Often, this list is too long, and will overlap
%% other information printed in the page headers. This command allows
%% the author to define a more concise list
%% of authors' names for this purpose.
\renewcommand{\shortauthors}{Cusati and Brown}

%%
%% The abstract is a short summary of the work to be presented in the
%% article.
\begin{abstract}
Software engineering research has experienced sustained growth in both output and participation over the past decades. Yet concerns persist about the field's ability to accumulate, integrate, and reuse knowledge in ways that support long-term progress. To better understand how the community itself perceives these challenges, we analyze responses from the ICSE 2026 Future of Software Engineering pre-survey, which captures perspectives from a globally distributed and highly experienced set of researchers. Our analysis reveals a tension between increasing research productivity and the limited mechanisms available for synthesizing results, tracking evolving claims, and supporting cumulative understanding over time.

Building on these observations, we diagnose four interrelated structural breakdowns: papers function as isolated knowledge units with claims embedded in prose; context and provenance are lost as knowledge moves through the publication pipeline; claims evolve without systematic tracking; and incentive structures favor novelty over consolidation. We argue that addressing these barriers requires rethinking the fundamental properties of research artifacts.

We articulate four technology-agnostic principles for future research artifacts: structured and interpretable representations of claims and evidence; inspectable and provenance-aware documentation of methodological decisions; long-lived and reusable substrates that evolve beyond publication; and governance mechanisms that align individual incentives with collective knowledge-building goals. We discuss implications for research practice, publication norms, and community infrastructure, positioning FOSE as a venue for experimenting with alternative artifact designs that support cumulative scientific progress.
\end{abstract}

%%
%% The code below is generated by the tool at http://dl.acm.org/ccs.cfm.
%% Please copy and paste the code instead of the example below.
%%
\begin{CCSXML}
<ccs2012>
 <concept>
  <concept_id>10011007.10011074.10011099.10011102.10011103</concept_id>
  <concept_desc>Software and its engineering~Software creation and management</concept_desc>
  <concept_significance>500</concept_significance>
 </concept>
 <concept>
  <concept_id>10011007.10011074.10011111.10011113</concept_id>
  <concept_desc>Software and its engineering~Empirical software validation</concept_desc>
  <concept_significance>500</concept_significance>
 </concept>
 <concept>
  <concept_id>10002944.10011123.10011131</concept_id>
  <concept_desc>General and reference~Empirical studies</concept_desc>
  <concept_significance>300</concept_significance>
 </concept>
</ccs2012>
\end{CCSXML}

\ccsdesc[500]{Software and its engineering~Software creation and management}
\ccsdesc[500]{Software and its engineering~Empirical software validation}
\ccsdesc[300]{General and reference~Empirical studies}

%%
%% Keywords. The author(s) should pick words that accurately describe
%% the work being presented. Separate the keywords with commas.
\keywords{knowledge accumulation, research artifacts, software engineering, cumulative progress, research infrastructure}

%% A "teaser" image appears between the author and affiliation
%% information and the body of the document, and typically spans the
%% page.
% \begin{teaserfigure}
%   \includegraphics[width=\textwidth]{sampleteaser}
%   \caption{Seattle Mariners at Spring Training, 2010.}
%   \Description{Enjoying the baseball game from the third-base
%   seats. Ichiro Suzuki preparing to bat.}
%   \label{fig:teaser}
% \end{teaserfigure}

%%
%% This command processes the author and affiliation and title
%% information and builds the first part of the formatted document.
\maketitle

\section{Introduction}
\label{sec:introduction}

Software engineering research has experienced remarkable growth over the past decades. Conference submissions have increased, publication venues have proliferated, and the community has become truly global. By many measures, the field is thriving: researchers are productive, techniques are advancing, and new areas of inquiry continue to emerge. Yet beneath this surface of activity lies a persistent challenge: the difficulty of building cumulative knowledge that connects, consolidates, and extends prior work in ways that support long-term scientific progress.

This challenge is not new, nor is it unique to software engineering. Across many scientific disciplines, researchers have observed that increasing publication rates do not automatically translate into deeper understanding or more integrated knowledge~\cite{ioannidis2005,fortunato2018}. In software engineering specifically, concerns about fragmentation, replication, and the ability to synthesize results across studies have been raised repeatedly~\cite{shull2008,sjberg2007}. Despite these concerns, the structural factors that limit cumulative progress---and the properties that future research artifacts might need to address them---remain underexplored.

This paper offers a community-informed perspective on knowledge accumulation challenges in software engineering research. We begin by examining responses to the ICSE 2026 Future of Software Engineering track pre-survey, which reveals a mature, productive, and globally distributed community. This demographic profile is significant: if knowledge accumulation remains limited despite high levels of expertise, experience, and output, then the barriers to progress are unlikely to be individual. They are structural---embedded in how research artifacts are designed, how knowledge is represented, and how incentives shape research practice.

Building on this observation, we diagnose four interrelated structural breakdowns that constrain cumulative knowledge building:
\begin{enumerate}
\item \textbf{Papers as isolated knowledge units:} Claims and evidence are embedded in prose, making direct comparison and synthesis difficult as the literature grows.
\item \textbf{Loss of context and provenance:} Motivation, assumptions, and methodological decisions fade over time, increasing the cost of building on prior work.
\item \textbf{Claims evolve without tracking:} Refinements, contradictions, and relationships between claims are documented in prose but not systematically tracked or resolved.
\item \textbf{Incentive structures favor novelty over accumulation:} Publication systems reward new contributions while undervaluing replication, consolidation, and infrastructure work.
\end{enumerate}

These breakdowns are not independent---they reinforce one another, creating a system in which knowledge accumulates slowly and inefficiently despite the community's best efforts.

In response, we articulate four principles for designing research artifacts that better support cumulative progress:
\begin{enumerate}
\item \textbf{Structured and interpretable:} Artifacts should make claims, evidence, and context explicit and directly accessible.
\item \textbf{Inspectable and provenance-aware:} Artifacts should preserve the full provenance of results and make it transparent.
\item \textbf{Long-lived and reusable:} Artifacts should support evolution and reuse beyond the moment of publication.
\item \textbf{Governed with human oversight:} Artifacts and infrastructures should be managed by community processes that ensure quality, integrity, and alignment with collective goals.
\end{enumerate}

These principles are intentionally technology-agnostic. We do not prescribe specific tools, platforms, or workflows. Instead, we describe properties that any system aiming to support knowledge accumulation should strive for, leaving room for diverse implementations across domains and research communities.

\subsection{Contributions}

This paper makes the following contributions:

\begin{itemize}
\item A community-informed diagnosis of structural barriers to knowledge accumulation in software engineering, grounded in evidence from the ICSE 2026 FOSE pre-survey.
\item An articulation of four interrelated breakdowns that explain why cumulative progress remains limited despite high levels of expertise and productivity.
\item Four principles for future research artifacts that address these breakdowns, presented as technology-agnostic guidelines rather than prescriptive solutions.
\item A discussion of implications for research practice, publication norms, community infrastructure, and the role of FOSE as an experimentation venue for alternative artifact types.
\end{itemize}

\subsection{Scope and Positioning}

This is not a tooling paper. We do not present a new system, platform, or methodology. Nor do we claim to have solved the challenges we diagnose. Instead, we offer a reflective, agenda-setting perspective appropriate for the FOSE track: a synthesis of community observations, a diagnosis of structural limitations, and a set of principles to guide future work. Our goal is to articulate why cumulative knowledge progress remains constrained and what properties future research artifacts might need to address these constraints.

We position this work as a call to rethink the fundamental design of research artifacts---not because current practices are wrong, but because they were optimized for different goals (dissemination, novelty, individual credit) and may be insufficient for the cumulative, long-term knowledge building that scientific progress requires.

\subsection{Paper Organization}

The remainder of this paper is organized as follows. Section~\ref{sec:background} situates our work within related perspectives on knowledge accumulation, reproducibility, and research infrastructure. Section~\ref{sec:survey} presents findings from the ICSE 2026 FOSE pre-survey, establishing the community profile that motivates our structural diagnosis. Section~\ref{sec:breakdown} diagnoses four interrelated breakdowns that limit cumulative progress. Section~\ref{sec:principles} articulates four principles for future research artifacts. Section~\ref{sec:implications} discusses implications for research practice, publication, and infrastructure. Section~\ref{sec:limitations} acknowledges limitations and open questions. Section~\ref{sec:conclusion} concludes with reflections on the path forward.

% Removed standalone background section - merged essential citations into introduction
% \section{Background and Related Perspectives}
\label{sec:background}

The challenges of cumulative knowledge building are not unique to software engineering, nor are they new. Across scientific disciplines, researchers and meta-researchers have documented tensions between publication growth and knowledge synthesis, raised concerns about reproducibility and replication, and called for better infrastructure to support long-term scientific progress. This section situates our work within these broader conversations.

\subsection{Knowledge Accumulation in Science}

The challenge of accumulating knowledge has been a recurring theme in philosophy of science and meta-research. Kuhn's notion of paradigm shifts highlighted periods of consolidation and synthesis~\cite{kuhn1962}. More recently, concerns about the ``reproducibility crisis'' have drawn attention to the difficulty of verifying and building on prior results~\cite{ioannidis2005,baker2016}. These discussions emphasize that scientific progress depends not only on producing new findings but also on mechanisms for validating, integrating, and reusing prior work.

In software engineering, similar concerns have been raised. Studies have documented challenges in replication~\cite{juristo2011,herman2020}, challenges in synthesizing results across empirical studies~\cite{kitchenham2007}, and the difficulty of comparing techniques when methodologies and datasets vary~\cite{basili1999}. Researchers have also noted that the field's rapid growth, while positive in many ways, has made it harder to maintain shared understanding and cumulative building across subfields~\cite{sjberg2007}.

\subsection{Research Infrastructure and Artifacts}

A growing body of work has explored how research infrastructure can support cumulative progress. Cyberinfrastructure initiatives~\cite{atkins2003} have emphasized the need for shared computational resources, data repositories, and collaborative platforms. In software engineering specifically, efforts like artifact evaluation tracks~\cite{herman2020}, benchmark repositories~\cite{do2005}, and replication packages~\cite{gonzalez-barahona2012} aim to make research more transparent and reusable.

Beyond software engineering, knowledge graph initiatives in other scientific domains offer relevant examples. The Open Research Knowledge Graph (ORKG)~\cite{orkg} structures contributions from scientific papers as semantic triples, enabling comparison and aggregation across studies. Similar efforts in scientific knowledge extraction~\cite{luan2024} and academic graph construction~\cite{wang2020} aim to represent scholarly knowledge in machine-readable, queryable forms. These systems demonstrate the potential of structured representations to support synthesis, but they also highlight challenges: creating and maintaining such infrastructures requires sustained effort, community coordination, and alignment of incentives~\cite{collberg2016}.

\subsection{Incentive Structures and Community Practices}

Several researchers have examined how incentive structures shape scientific practice. Publication pressures, the emphasis on novelty in review processes, and the undervaluation of negative results and replication have all been identified as barriers to cumulative progress~\cite{smaldino2016,nosek2012}. In software engineering, Shull et al.~\cite{shull2008} discussed the role of community infrastructure and incentives in supporting empirical research. More recent work has called for recognizing contributions to datasets, benchmarks, and infrastructure as valuable scholarly outputs~\cite{allen2019}.

The tension between individual incentives (publishing novel results to advance one's career) and collective needs (consolidating knowledge, maintaining infrastructure) is a well-recognized collective action problem~\cite{osterloh2020}. Addressing it requires not only technical solutions but also changes in how contributions are evaluated, how credit is assigned, and how community resources are governed.

\subsection{Meta-Research and Science of Science}

The emerging field of meta-research, or the ``science of science,'' studies the processes and structures that shape scientific work~\cite{ioannidis2018,fortunato2018}. This research has documented patterns in citation networks, collaboration structures, and the lifecycle of scientific ideas. It has also examined how publication systems, peer review, and funding mechanisms influence what research gets done and how it is communicated~\cite{wang2013}.

One relevant finding is that scientific communities differ in how well they support cumulative building. Fields with well-established benchmarks, standardized evaluation protocols, and shared datasets often exhibit faster cumulative progress than fields where each study uses custom methodologies~\cite{donoho2017}. This suggests that infrastructure for comparison and reuse can have measurable impacts on the pace of knowledge accumulation.

\subsection{Positioning This Work}

Our work builds on these perspectives but focuses specifically on structural factors that limit cumulative progress in software engineering research. We differ from prior work in three ways:

First, we ground our analysis in community perspectives gathered through the ICSE 2026 FOSE pre-survey. Rather than relying solely on bibliometric analysis or case studies, we use survey data to establish that the community itself recognizes accumulation challenges and that these challenges persist despite high levels of expertise and productivity.

Second, we provide a structured diagnosis of four interrelated breakdowns, connecting individual challenges (isolated papers, lost context, untracked claims, misaligned incentives) into a coherent account of why progress remains constrained.

Third, we articulate technology-agnostic principles for future research artifacts rather than proposing specific tools or platforms. Our goal is not to advocate for any particular system but to describe properties that any infrastructure aiming to support cumulative progress should strive for. This framing is intentionally broad, allowing for diverse implementations across different subfields and research communities.

We do not claim that these challenges are unique to software engineering, nor that solutions developed in other fields are irrelevant. Instead, we offer a community-informed synthesis tailored to the software engineering context, with the hope that the principles we articulate can inform both local efforts within the field and broader conversations about the future of scientific infrastructure.

\section{Community Signals from the ICSE 2026 FOSE Pre-Survey}
\label{sec:survey}

To ground our discussion in community perspectives, we draw on responses from the ICSE 2026 Future of Software Engineering track pre-survey. This survey was designed to understand how members of the software engineering research community perceive current challenges and opportunities in the field. We received 280 completed responses from researchers representing diverse roles, experience levels, and geographic regions.

Our purpose in examining this survey is not to conduct a comprehensive empirical study, but rather to establish context and legitimacy for the structural challenges we diagnose in this paper. The survey data reveals a community profile that is critical to our central argument: if knowledge accumulation remains limited despite significant expertise and participation, then the barriers must be structural rather than individual.

\subsection{Community Profile}

The survey respondents represent a mature, active, and globally distributed research community. Table~\ref{tab:survey-demographics} summarizes key demographic characteristics.

\begin{table}[t]
\caption{FOSE 2026 pre-survey respondent demographics (n=280)}
\label{tab:survey-demographics}
\begin{tabular}{lrr}
\toprule
\textbf{Characteristic} & \textbf{Count} & \textbf{Percentage} \\
\midrule
\multicolumn{3}{l}{\textit{Years in SE Research Community}} \\
\quad 0--1 years & 12 & 4.3\% \\
\quad 2--3 years & 26 & 9.3\% \\
\quad 4--5 years & 29 & 10.4\% \\
\quad 6--10 years & 50 & 17.9\% \\
\quad 11--20 years & 82 & 29.3\% \\
\quad 21+ years & 78 & 27.9\% \\
\midrule
\multicolumn{3}{l}{\textit{Papers Authored (Past 3 Years)}} \\
\quad 0 papers & 8 & 2.9\% \\
\quad 1--3 papers & 54 & 19.3\% \\
\quad 4--10 papers & 91 & 32.5\% \\
\quad 11--20 papers & 73 & 26.1\% \\
\quad 21--40 papers & 32 & 11.4\% \\
\quad 41+ papers & 19 & 6.8\% \\
\midrule
\multicolumn{3}{l}{\textit{Geographic Distribution}} \\
\quad Europe & 139 & 49.6\% \\
\quad North America & 70 & 25.0\% \\
\quad Asia & 41 & 14.6\% \\
\quad South America & 14 & 5.0\% \\
\quad Oceania & 11 & 3.9\% \\
\quad Middle East & 2 & 0.7\% \\
\bottomrule
\end{tabular}
\end{table}

\textbf{Experience.} The majority of respondents (57.1\%) have been involved in software engineering research for over a decade, with 29.3\% reporting 11--20 years of experience and 27.9\% reporting 21 or more years. This is not a community of newcomers struggling to establish research programs, but rather one populated by seasoned researchers who have observed and participated in the field's evolution over substantial periods.

\textbf{Productivity.} The community is actively engaged in producing research outputs. Nearly half of respondents (44.3\%) reported authoring 11 or more paper submissions in the past three years alone---an average of at least 3--4 papers per year. An additional 32.5\% reported 4--10 submissions. Only 2.9\% reported zero submissions. This high level of productivity suggests a community that is not constrained by lack of effort or output capacity.

\textbf{Global Reach.} Respondents represent six geographic regions spanning Europe, North and South America, Asia, Oceania, and the Middle East. While Europe is most heavily represented (49.6\%), North American (25.0\%) and Asian (14.6\%) researchers form substantial communities. This global distribution indicates that the challenges we discuss are not artifacts of a single research culture or institutional context, but rather reflect patterns that transcend geographic boundaries.

\subsection{Implications for Knowledge Accumulation}

The demographic profile of survey respondents carries a critical implication for our argument. The software engineering research community is not lacking in \textit{expertise}---over half the respondents have more than a decade of experience. It is not lacking in \textit{participation}---nearly half produce at least 11 paper submissions every three years. And it is not limited to a narrow geographic or cultural context---six regions are represented, with substantial participation from multiple continents.

If knowledge accumulation remains fragmented, if claims and evidence remain weakly linked, and if rediscovery costs remain high despite this level of experience, productivity, and global engagement, then the barriers to cumulative progress must be \textit{structural}. They cannot be attributed to insufficient expertise, inadequate effort, or limited participation. Instead, they must arise from the ways in which research artifacts are produced, represented, and connected to one another over time.

\subsection{Recognized Challenges}

Beyond demographics, the survey responses reveal that the community itself recognizes systemic challenges. When asked what aspects of the software engineering research community do not work well, respondents frequently pointed to issues related to knowledge synthesis and cumulative building:

\begin{quote}
\textit{``Reviewing process: quality of reviews is not always good---too many papers get submitted (and resubmitted).''}
\end{quote}

\begin{quote}
\textit{``We're not inclusive or sustainable enough insofar as the remote participation is lousy. I'd do away with all of our program committees and just do journal first for everything. I'd like to see people have more flexibility in which and whether to attend conferences and less waste and resubmission and re-review of papers.''}
\end{quote}

\begin{quote}
\textit{``Adopting new technologies. Let's say NLP community already adopted LLM based annotation with human supervision... yet in SE papers 30\% of the justification goes to why this technology works well. It's like doing two research in a single paper.''}
\end{quote}

These comments hint at deeper structural issues: cycles of resubmission that fragment results across venues and versions, difficulties in building on prior work efficiently, and challenges in tracking how ideas evolve as they move through the publication pipeline. Respondents also expressed concerns about review quality, the stress of managing high submission volumes, and the difficulty of preparing mentees for a system that rewards novelty over consolidation.

\subsection{Transition to Structural Diagnosis}

The survey data establishes that the software engineering research community possesses significant expertise, maintains high productivity, and operates across global contexts. Yet respondents themselves recognize persistent challenges in how knowledge is synthesized, reviewed, and built upon. These observations set the stage for a more detailed examination of the structural barriers that limit cumulative knowledge accumulation---the subject of the next section.

\section{Where Knowledge Accumulation Breaks Down}
\label{sec:breakdown}

If the software engineering community is experienced, productive, and globally distributed, yet faces persistent challenges synthesizing results and building on prior work, then the barriers must be structural. We diagnose four interrelated breakdowns arising from how research artifacts are currently produced, represented, and connected.

\subsection{Papers as Isolated Knowledge Units}

Research papers package claims, evidence, and context into narrative prose optimized for dissemination. While effective for presenting new ideas, this format creates barriers to cumulative building. Claims are embedded in paragraphs, interleaved with motivation and methods, making extraction and comparison across papers require reading entire documents and reconciling terminological differences. Evidence linkages remain implicit---readers must infer which results support which claims under what assumptions. Even when artifacts are shared, results are static at publication time; updates to datasets or baselines require new papers citing the old, while original claims remain unchanged. The consequence is fragmentation: each paper is an island connected only through citations and prose, requiring repeated manual synthesis that does not scale as literature grows.

\subsection{Loss of Context and Provenance}

Research involves countless contextual decisions: which dataset to use, how to split data, which baselines to compare, how to handle edge cases. Papers describe these choices but often omit the rationale due to space constraints. As work is cited and summarized, motivation and assumptions fade---carefully qualified findings become unqualified facts in subsequent literature. Methodological decisions must be reverse-engineered, sometimes revealing that subtle choices significantly impacted results. Papers present polished final versions, hiding the evolutionary path from hypothesis to result. When context and provenance are lost, later researchers must either accept prior work at face value or invest substantial effort reconstructing reasoning, slowing progress and increasing misapplication risk.

\subsection{Claims Evolve Without Tracking}

Scientific claims are refined, qualified, contradicted, and superseded as evidence accumulates, yet the publication system provides limited tracking mechanisms. The literature may contain conflicting claims---technique A outperforms B in one paper, the reverse in another---but contradictions are not systematically flagged. Researchers discovering conflicts must investigate causes themselves, often finding subtle methodological differences account for divergence. Refinements are implicit: a follow-on paper may qualify a prior claim, but the original remains unchanged. Claim relationships---does one extend, contradict, or depend on another?---are expressed only in natural language like "building on [12]" or "in contrast to [34]." Without structured representations, knowledge remains fragmented and ambiguous, requiring extensive manual synthesis to determine what is currently believed, under what conditions, and with what confidence.

\subsection{Incentive Structures Favor Novelty Over Accumulation}

The final breakdown is embedded in incentive structures. Publication venues, hiring criteria, and funding mechanisms reward novelty and originality. Conference and journal reviews prioritize new techniques and findings; replication studies, negative results, and syntheses face higher acceptance bars even when they would advance collective understanding. Researchers investing in replication, dataset curation, or shared infrastructure face opportunity costs---these efforts may not yield top-venue publications or count heavily in promotion reviews. Building knowledge repositories, ontologies, and interoperability layers requires sustained effort often led by small groups without commensurate recognition. The result is a collective action problem: everyone benefits from better infrastructure, but individuals are disincentivized from contributing. As long as novelty is privileged over accumulation, knowledge remains fragmented and progress constrained.

These four breakdowns reinforce one another, creating a system where knowledge accumulates slowly despite community expertise and productivity. Incremental fixes---better citation practices, more replications, improved repositories---are necessary but insufficient. Addressing these barriers requires rethinking the fundamental properties of research artifacts themselves.

\section{Rethinking Research Artifacts for Cumulative Progress}
\label{sec:principles}

The structural barriers diagnosed in the previous section are not inevitable. They arise from design choices embedded in how research artifacts are currently produced and shared. If papers-as-documents create fragmentation, and if current incentive structures discourage consolidation, then addressing these challenges requires reconsidering the fundamental properties that research artifacts should have. In this section, we articulate four principles to guide the design of artifacts that support cumulative, interpretable, and reusable knowledge building.

These principles are intentionally abstract. We do not prescribe specific technologies, platforms, or workflows. Instead, we describe properties that any system, tool, or practice aiming to support knowledge accumulation should strive for. Concrete implementations will vary across domains and communities, but the underlying principles provide a shared foundation for evaluating and improving research infrastructure.

\subsection{Principle 1: Structured and Interpretable}

\textbf{Principle:} Research artifacts should make claims, evidence, and context explicit and directly accessible, not only embedded in prose.

The first breakdown---papers as isolated units---stems from the fact that claims and evidence are woven into narrative text. While prose is valuable for explaining motivation and argumentation, it is not optimal for supporting cumulative synthesis. A reader wishing to compare claims across papers must extract and interpret statements from natural language, a process that does not scale as the literature grows.

\textbf{What this principle requires.} Artifacts should represent claims, evidence, and context as structured, first-class entities. A claim should be identifiable: ``Technique X improves metric Y on dataset Z by $\delta$ under conditions C.'' The evidence supporting that claim---experimental configuration, data, results, analysis---should be explicitly linked. The context---assumptions, limitations, methodological decisions---should be preserved alongside the claim, not buried in prose.

Structured representations enable direct comparison, aggregation, and reasoning over results. If two papers make conflicting claims about the same technique, the conflict becomes visible and resolvable by examining the structured evidence and context. If a dataset is updated or a metric refined, claims referencing that dataset or metric can be systematically identified and reevaluated.

\textbf{Instantiation examples.} Structured representations could take many forms:
\begin{itemize}
\item \textit{Semantic annotations:} Papers supplemented with machine-readable metadata describing claims, experimental setups, and results.
\item \textit{Knowledge graph representations:} Claims, techniques, datasets, and metrics represented as entities with typed relationships, enabling queries like ``Which papers claim improvements on dataset D using technique family T?''
\item \textit{Computational notebooks with assertions:} Executable analyses in which key claims are formalized as assertions that can be tested, versioned, and reused.
\end{itemize}

These are illustrative, not prescriptive. The principle is that structure should complement prose, making knowledge directly accessible for synthesis while preserving the explanatory power of narrative.

\subsection{Principle 2: Inspectable and Provenance-Aware}

\textbf{Principle:} Research artifacts should preserve the full provenance of claims---from raw data through methodological decisions to final results---and make this provenance inspectable.

The second breakdown---loss of context and provenance---occurs because the reasoning behind decisions fades as knowledge moves through the publication pipeline. A paper reports final results but often omits the path taken to reach them: why this dataset, why this baseline, why this evaluation protocol. When later researchers attempt to build on the work, they must reconstruct this reasoning, often discovering that subtle choices had significant consequences.

\textbf{What this principle requires.} Artifacts should document not only what was found, but how and why. Every claim should trace back to its sources: the data, the code, the configuration, the assumptions. Methodological decisions should be explicitly recorded and justified: ``We used dataset D because it contains feature F relevant to hypothesis H.'' When results are updated---due to new data, corrected analyses, or refined methods---the provenance chain should track these changes, preserving the history of how understanding evolved.

Provenance enables trust and reproducibility. A researcher can inspect the lineage of a claim, verify that it rests on sound evidence, and understand the conditions under which it holds. When claims conflict, provenance helps diagnose the source of divergence: different data, different preprocessing, different evaluation criteria.

\textbf{Instantiation examples.} Provenance tracking could be realized through:
\begin{itemize}
\item \textit{Versioned computational artifacts:} Code, data, and configurations stored in version control systems with clear lineage from raw inputs to published results.
\item \textit{Provenance graphs:} Explicit representations of how results were derived, linking claims to experiments, experiments to data, and data to sources.
\item \textit{Decision logs:} Structured records of methodological choices, their rationales, and their impacts, made part of the permanent artifact.
\end{itemize}

Again, these are examples. The principle is that artifacts should be inspectable---transparent about their origins and decisions---so that future work can be built on solid, well-understood foundations.

\subsection{Principle 3: Long-Lived and Reusable}

\textbf{Principle:} Research artifacts should support evolution and reuse, not remain static at the moment of publication.

The third breakdown---claims evolve without tracking---and aspects of the first breakdown arise because papers are snapshots frozen in time. Once published, a paper does not update when new evidence emerges, when datasets are revised, or when methods are improved. Follow-on work cites the paper and describes differences in prose, but the original claim remains unchanged in the literature.

\textbf{What this principle requires.} Artifacts should be living substrates that can be updated, extended, and reused. When a dataset is corrected, claims depending on that dataset should be re-evaluable, not obsolete. When a technique is refined, prior results should be comparable to new results using updated methods. When claims are qualified or contradicted, these relationships should be reflected in the artifact, not scattered across disconnected papers.

This does not mean papers should be continuously rewritten. Rather, the underlying structured representations---claims, evidence, configurations---should be decoupled from narrative documents. Narratives can remain stable as historical records, while the structured substrates evolve as understanding progresses.

\textbf{Instantiation examples.} Long-lived artifacts could include:
\begin{itemize}
\item \textit{Living knowledge bases:} Repositories where claims, datasets, and techniques are maintained, versioned, and updated as new evidence accumulates.
\item \textit{Executable benchmarks:} Evaluation frameworks that can be rerun with updated data or baselines, allowing claims to be retested and results to be continuously refined.
\item \textit{Modular artifact ecosystems:} Components (datasets, models, evaluation scripts) designed for composition and reuse, enabling new studies to build directly on prior artifacts rather than reimplementing from scratch.
\end{itemize}

The principle is that knowledge should outlive individual papers, accumulating in shared substrates that reduce redundancy and enable direct building.

\subsection{Principle 4: Governed with Human Oversight}

\textbf{Principle:} Research artifacts and infrastructures should be governed by community processes that ensure quality, integrity, and ethical responsibility.

The fourth breakdown---incentive structures favor novelty---reflects the tension between individual incentives and collective needs. Even if we had perfect technical infrastructure, cumulative progress requires coordination: standards for quality, mechanisms for resolving conflicts, and recognition for contributions that consolidate rather than merely add.

\textbf{What this principle requires.} Artifacts cannot govern themselves. Structured representations, provenance tracking, and living substrates are valuable only if the community trusts them, maintains them, and uses them responsibly. This requires human oversight: peer review for knowledge contributions, curation of shared resources, community processes for resolving disputes or establishing standards, and credit systems that value infrastructure work alongside novel research.

Governance also addresses ethical concerns. Structured knowledge artifacts raise questions about ownership, attribution, bias, and access. Who controls a shared knowledge base? How is credit assigned when multiple researchers contribute incrementally? How do we ensure that knowledge infrastructures do not perpetuate biases present in underlying data? These are social and ethical questions, not purely technical ones, and they require community deliberation and ongoing stewardship.

\textbf{Instantiation examples.} Governance mechanisms could include:
\begin{itemize}
\item \textit{Community curation processes:} Peer-reviewed contributions to shared knowledge bases, with clear criteria for inclusion, updating, and deprecation.
\item \textit{Credit and attribution systems:} Mechanisms that recognize contributions to infrastructure, replication, and consolidation, not only novel results.
\item \textit{Ethical oversight boards:} Community structures responsible for addressing bias, fairness, and access issues in shared knowledge infrastructures.
\item \textit{Transparent governance models:} Decision-making processes for managing shared resources, resolving conflicts, and setting standards, with broad community input.
\end{itemize}

The principle is that cumulative progress is a collective endeavor. Technical solutions alone are insufficient; they must be embedded in social practices that align individual incentives with collective goals and ensure responsible stewardship of shared knowledge.

\subsection{Mapping Principles to Breakdowns}

These four principles are not independent---they address the interconnected breakdowns diagnosed earlier:

\begin{itemize}
\item \textbf{Structured and interpretable} artifacts address the isolation of knowledge by making claims and evidence directly comparable and composable.
\item \textbf{Inspectable and provenance-aware} artifacts address the loss of context by preserving the reasoning and decisions behind results.
\item \textbf{Long-lived and reusable} artifacts address the static nature of claims by enabling evolution and incremental refinement over time.
\item \textbf{Governed with human oversight} addresses incentive misalignment by establishing community processes that value consolidation and ensure responsible stewardship.
\end{itemize}

Together, they describe a vision for research artifacts that support cumulative knowledge building: artifacts that are not isolated documents but interconnected, evolving substrates governed by community practices that align individual contributions with collective progress.

\subsection{From Principles to Practice}

These principles are aspirational. Realizing them fully would require significant changes in how research is conducted, reviewed, published, and rewarded. We do not claim that these changes are easy or that any single intervention will suffice. But the principles provide a framework for evaluating incremental steps: Does a proposed tool, platform, or practice make artifacts more structured, more inspectable, more reusable, or better governed? If so, it moves the community toward more cumulative progress.

Importantly, these principles are not tied to specific technologies. Knowledge graphs, computational notebooks, version control systems, and community repositories are all potential instantiations---but the principles themselves are technology-agnostic. They describe \textit{properties} that artifacts should have, leaving room for diverse implementations across different domains and research communities.

The challenge is to move from principles to practice: to design systems, establish norms, and reform incentives in ways that make cumulative knowledge building not only possible but rewarded. We turn to the broader implications of this shift in the next section.

\section{Implications for the Future of Software Engineering}
\label{sec:implications}

The principles articulated in the previous section are aspirational. Realizing them would require changes not only in how research artifacts are designed, but also in how research is conducted, reviewed, published, and rewarded. In this section, we discuss implications for research practice, publication norms, community infrastructure, and the role of FOSE as a venue for experimentation.

\subsection{Implications for Research Practice}

If research artifacts were structured, inspectable, provenance-aware, and long-lived, how would research practice change?

\textbf{Planning and documentation.} Researchers would need to document not only final results but also the process by which those results were obtained. This includes recording methodological decisions, their rationales, and their impacts. Tools like computational notebooks, version control systems, and workflow management platforms could support this documentation, but the cultural shift---valuing process alongside outcomes---is equally important.

\textbf{Structuring contributions.} Instead of presenting results solely in prose, researchers would also produce structured representations of claims, evidence, and context. This does not mean abandoning narrative---papers would still tell stories, explain motivations, and argue for significance. But they would be supplemented by machine-readable metadata, semantic annotations, or knowledge graph entries that make key contributions directly accessible for synthesis and reuse.

\textbf{Designing for reuse.} Researchers would consider reusability at the outset, not as an afterthought. Datasets would be documented with schemas and provenance. Code would be modular and well-documented. Evaluation protocols would be reproducible. The goal would be to produce artifacts that others can directly build on, reducing the need for reimplementation and lowering the barrier to cumulative progress.

\textbf{Collaborating across studies.} Structured artifacts enable new forms of collaboration. Researchers could contribute incrementally to shared knowledge bases, updating claims as new evidence emerges, proposing refinements to existing techniques, or resolving contradictions by comparing structured evidence. This would shift the unit of contribution from the paper to the structured artifact, enabling continuous, distributed knowledge building.

These changes would require time and effort. But if the infrastructure and incentives were in place, they could become standard practice, much as sharing code and data have become increasingly expected in many research communities.

\subsection{Implications for Publication and Review}

If the principles were adopted, what would change in how research is published and reviewed?

\textbf{Evaluating contributions beyond novelty.} Review processes would need to recognize contributions that consolidate, synthesize, or replicate prior work. A paper that resolves contradictions in the literature, curates a benchmark dataset, or provides infrastructure for cumulative building should be valued alongside papers that propose new techniques. This requires updating review criteria and training reviewers to assess different types of contributions.

\textbf{Artifact expectations.} Conferences and journals could establish expectations that submissions include not only papers but also structured artifacts: semantic annotations, provenance documentation, reusable components. Artifact evaluation tracks have made progress in this direction, but the principles suggest going further---treating structured artifacts as first-class contributions, not optional supplements.

\textbf{Living publications.} If artifacts evolve beyond publication, then papers could be viewed as snapshots of ongoing work rather than final records. A paper might introduce a technique and present initial results, while a living artifact continues to accumulate evidence, track refinements, and document extensions. This would require rethinking version control for scholarly communication and establishing norms for how to cite evolving artifacts.

\textbf{Community review and curation.} Governance mechanisms imply that some contributions would be reviewed not only at submission time but also post-publication, as they are integrated into shared knowledge bases. Community curation processes---similar to those used in some open-access repositories or collaborative platforms---could ensure that contributions meet quality standards, resolve conflicts, and maintain consistency over time.

These changes would be significant, but they are not without precedent. Fields like machine learning have experimented with leaderboards, benchmark repositories, and living evaluation frameworks. The principles suggest extending these experiments more broadly and embedding them in the publication process itself.

\subsection{Implications for Community Infrastructure}

If the principles were realized, what infrastructure would be needed?

\textbf{Shared knowledge repositories.} The community would need platforms for storing, querying, and updating structured research artifacts. These could take many forms: knowledge graphs representing claims and evidence; benchmark repositories hosting datasets and evaluation scripts; collaborative platforms where researchers propose and refine contributions. The key is that these repositories would be long-lived, maintained, and governed by community processes.

\textbf{Standards and interoperability.} For artifacts to be reusable across studies, the community would need shared standards: ontologies for representing claims and techniques, schemas for documenting datasets, protocols for reporting results. Establishing these standards requires coordination, but the benefits---reduced friction in synthesis and reuse---could be substantial.

\textbf{Credit and attribution systems.} Infrastructure contributions, replication efforts, and incremental refinements would need to be recognized and credited. This could involve alternative metrics (e.g., how often a dataset is reused, how many claims reference a benchmark), changes in tenure and promotion criteria, or platforms that track contributions beyond traditional papers. Aligning individual incentives with collective knowledge-building goals is essential for sustainability.

\textbf{Governance and stewardship.} Shared infrastructures require stewardship: maintaining repositories, resolving disputes, updating standards, addressing ethical concerns. The community would need governance structures---committees, boards, or distributed processes---to manage these responsibilities. Transparency, inclusivity, and accountability would be critical to ensuring that infrastructures serve the collective good.

Building and maintaining such infrastructure is not trivial. It requires sustained investment, coordination across institutions and subfields, and a willingness to prioritize collective needs alongside individual research agendas. But without infrastructure, the principles remain aspirational.

\subsection{FOSE as a Venue for Experimentation}

The Future of Software Engineering track is uniquely positioned to support experimentation with alternative artifact types and practices. FOSE papers do not need to conform to the same evaluation criteria as empirical studies or tool papers. They can be reflective, speculative, and forward-looking. This creates space for contributions that would not fit traditional tracks but that could inform the evolution of the field.

\textbf{Prototyping alternative formats.} FOSE could encourage submissions that experiment with structured artifacts, living documents, or collaborative knowledge-building platforms. These contributions would not be judged solely on novelty or empirical rigor but on their potential to demonstrate new ways of organizing and sharing knowledge.

\textbf{Community dialogues.} FOSE sessions could serve as forums for discussing infrastructure needs, governance models, and incentive alignment. The track could facilitate conversations that bridge research practice, publication systems, and community infrastructure---conversations that are difficult to have within the constraints of traditional paper sessions.

\textbf{Long-term tracking.} FOSE could track experiments over time, documenting what works, what fails, and what lessons emerge. This would provide evidence for evaluating alternative practices and informing broader community decisions about adopting new norms or infrastructures.

By embracing experimentation, FOSE can help the software engineering community move from principles to practice, testing ideas, refining approaches, and building momentum for structural change.

\subsection{Challenges and Trade-offs}

Realizing these implications would not be without challenges. Structured artifacts require effort to produce and maintain. Standards risk becoming rigid or exclusionary. Governance processes can be contentious. Infrastructure requires resources that may not be available or equitably distributed.

Moreover, there are trade-offs. Emphasizing structure and reusability could discourage exploratory, early-stage work that does not yet fit established frameworks. Focusing on consolidation could slow the introduction of genuinely novel ideas. Balancing these tensions---supporting both exploration and consolidation, novelty and cumulation, individual creativity and collective building---is an ongoing challenge.

The goal is not to replace current practices wholesale but to expand the range of valued contributions and supported practices. Papers will remain important for storytelling and argumentation. But they could be complemented by structured artifacts, living repositories, and community infrastructures that make cumulative progress more feasible and more rewarded.

\section{Limitations}
\label{sec:limitations}

This paper diagnoses structural barriers and proposes principles for future research artifacts, but does not provide complete solutions. We acknowledge several limitations.

\textbf{Survey limitations.} Our analysis draws on 280 FOSE pre-survey responses, representing a small, self-selected fraction of the global community. The data are descriptive, not causal---they establish that the community is experienced and productive, supporting our argument that barriers are structural, but do not prove specific breakdowns. The survey reflects perspectives at one moment; community concerns evolve over time. It also only reflects researcher insights focused on software engineering, and our proposed principles may not directly generalize to other research domains.

\textbf{Feasibility challenges.} Realizing these principles faces practical barriers. Producing structured, provenance-aware artifacts requires effort that researchers may not have without reduced demands or increased support. The needed infrastructure---shared repositories, standards, governance---does not yet exist in many areas and requires sustained investment and coordination~\cite{herman2020}. Adoption depends on changing incentives at multiple levels (\eg funding agencies, universities, conferences), which is difficult and slow. Cultural norms resist change, especially from those who succeed under current systems.

\textbf{Ethical concerns.} Structured knowledge infrastructures raise questions about ownership (\eg who controls shared knowledge bases?), bias (\eg artifacts may perpetuate creators' biases if diversity is lacking), access (\eg resources may concentrate in well-funded institutions), and privacy (\eg structured data about research processes requires ethical use and consent). These are not purely technical issues---they require community deliberation, ethical oversight, and ongoing stewardship. Any future infrastructure must address these concerns.

\section{Conclusion}
\label{sec:conclusion}

Software engineering research is productive and globally distributed, yet knowledge accumulation lags behind knowledge production. Leveraging insights from survey responses from the software engineering research community, this paper argues that the barriers are structural: papers function as isolated documents, provenance is lost, claims evolve without tracking, and incentives favor novelty over consolidation. Addressing these barriers requires rethinking research artifacts themselves. We have proposed four principles---\allowbreak structured and interpretable, inspectable and provenance-aware, long-lived and reusable, and governed with human oversight---to guide future work.

These principles are aspirational and technology-agnostic. Realizing them requires changes in practice, publication norms, infrastructure, and incentives. The challenges are significant, but the alternative---continuing with practices optimized for dissemination rather than accumulation---will perpetuate fragmentation as the field grows.

The Future of Software Engineering track provides space for this conversation. Next steps include experimenting with alternative artifact designs, developing infrastructure aligned with these principles, and revising recognition systems to value consolidation alongside novelty. Most importantly, the community must engage in dialogue about what cumulative progress requires and what trade-offs are acceptable.

The future of software engineering depends not only on what we discover but on how we organize and preserve what we know. Building infrastructures that support cumulative knowledge building---making it rewarded, not just possible---is essential to ensuring the next generation inherits not just a literature to read but a knowledge base to build upon.


%%
%% The next two lines define the bibliography style to be used, and
%% the bibliography file.
\bibliographystyle{ACM-Reference-Format}
\bibliography{bibliography,software}

%%
%% If your work has an appendix, this is the place to put it.
% \appendix

% \section{Research Methods}

\end{document}
