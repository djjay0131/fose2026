\section{Limitations}
\label{sec:limitations}

This paper diagnoses structural barriers and proposes principles for future research artifacts, but does not provide complete solutions. We acknowledge several limitations.

\textbf{Survey limitations.} Our analysis draws on 280 FOSE pre-survey responses, representing a small, self-selected fraction of the global community. The data are descriptive, not causal---they establish that the community is experienced and productive, supporting our argument that barriers are structural, but do not prove specific breakdowns. The survey reflects perspectives at one moment; community concerns evolve over time. It also only reflects researcher insights focused on software engineering, and our proposed principles may not directly generalize to other research domains.

\textbf{Feasibility challenges.} Realizing these principles faces practical barriers. Producing structured, provenance-aware artifacts requires effort that researchers may not have without reduced demands or increased support. The needed infrastructure---shared repositories, standards, governance---does not yet exist in many areas and requires sustained investment and coordination~\cite{herman2020}. Adoption depends on changing incentives at multiple levels (\eg funding agencies, universities, conferences), which is difficult and slow. Cultural norms resist change, especially from those who succeed under current systems.

\textbf{Ethical concerns.} Structured knowledge infrastructures raise questions about ownership (\eg who controls shared knowledge bases?), bias (\eg artifacts may perpetuate creators' biases if diversity is lacking), access (\eg resources may concentrate in well-funded institutions), and privacy (\eg structured data about research processes requires ethical use and consent). These are not purely technical issues---they require community deliberation, ethical oversight, and ongoing stewardship. Any future infrastructure must address these concerns.
