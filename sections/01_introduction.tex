\section{Introduction}
\label{sec:introduction}

Software engineering research is thriving by most measures. Conference submissions keep rising, publication venues have multiplied, and the community has gone global. Researchers are productive, techniques keep advancing, and new areas of inquiry emerge every year. But a persistent challenge lurks beneath all this activity: we struggle to build cumulative knowledge that actually connects, consolidates, and extends prior work.

This is not a new concern, nor one unique to our field. Across science, more publications have not automatically meant deeper understanding~\cite{ioannidis2005,fortunato2018}. In software engineering specifically, fragmentation and the difficulty of synthesizing results have been raised repeatedly~\cite{cruzes2010synthesizing,shull2008,sjberg2007}. Yet the structural factors limiting cumulative progress remain underexplored.

This paper offers a community-informed perspective on knowledge accumulation in software engineering. Drawing on 280 ICSE 2026 FOSE pre-survey responses from an experienced (57\% with 10+ years), productive (44\% with 11+ papers in 3 years), globally distributed community, we diagnose four structural breakdowns: papers as isolated units, lost context and provenance, untracked claim evolution, and incentives favoring novelty over consolidation. We then propose four principles for research artifacts: structured and interpretable, inspectable and provenance-aware, long-lived and reusable, and community-governed. We do not prescribe specific tools or claim to solve these challenges; instead, we synthesize community observations into a diagnosis and principles to guide future work.
