\section{Introduction}
\label{sec:introduction}

Software engineering research has experienced remarkable growth over the past decades. Conference submissions have increased, publication venues have proliferated, and the community has become truly global. By many measures, the field is thriving: researchers are productive, techniques are advancing, and new areas of inquiry continue to emerge. Yet beneath this surface of activity lies a persistent challenge: the difficulty of building cumulative knowledge that connects, consolidates, and extends prior work in ways that support long-term scientific progress.

This challenge is not new, nor is it unique to software engineering. Across many scientific disciplines, researchers have observed that increasing publication rates do not automatically translate into deeper understanding or more integrated knowledge~\cite{ioannidis2005,fortunato2018}. In software engineering specifically, concerns about fragmentation, replication, and the ability to synthesize results across studies have been raised repeatedly~\cite{cruzes2010synthesizing,shull2008,sjberg2007}. Despite these concerns, the structural factors that limit cumulative progress---and the properties that future research artifacts might need to address them---remain underexplored.

This position paper offers a community-informed perspective on knowledge accumulation challenges in software engineering. Grounded in responses from 280 ICSE 2026 FOSE pre-survey participants\allowbreak---an experienced (57\% with 10+ years), productive (44\% with 11+ papers in 3 years), globally distributed community---we diagnose four interrelated structural breakdowns: papers as isolated units, lost context and provenance, untracked claim evolution, and misaligned incentives favoring novelty over consolidation. In response, we propose four technology-agnostic design principles for research artifacts: structured and interpretable representations, inspectable and provenance-aware documentation, long-lived and reusable substrates, and community-governed infrastructures. This FOSE contribution synthesizes community observations into a diagnosis of structural limitations and principles to guide future work, without prescribing specific tools or claiming to solve the challenges we identify.
