\section{Introduction}
\label{sec:introduction}

Software engineering research has experienced remarkable growth over the past decades. Conference submissions have increased, publication venues have proliferated, and the community has become truly global. By many measures, the field is thriving: researchers are productive, techniques are advancing, and new areas of inquiry continue to emerge. Yet beneath this surface of activity lies a persistent challenge: the difficulty of building cumulative knowledge that connects, consolidates, and extends prior work in ways that support long-term scientific progress.

This challenge is not new, nor is it unique to software engineering. Across many scientific disciplines, researchers have observed that increasing publication rates do not automatically translate into deeper understanding or more integrated knowledge~\cite{ioannidis2005,fortunato2018}. In software engineering specifically, concerns about fragmentation, replication, and the ability to synthesize results across studies have been raised repeatedly~\cite{shull2008,sjberg2007}. Despite these concerns, the structural factors that limit cumulative progress---and the properties that future research artifacts might need to address them---remain underexplored.

This paper offers a community-informed perspective on knowledge accumulation challenges in software engineering research. We begin by examining responses to the ICSE 2026 Future of Software Engineering track pre-survey, which reveals a mature, productive, and globally distributed community. This demographic profile is significant: if knowledge accumulation remains limited despite high levels of expertise, experience, and output, then the barriers to progress are unlikely to be individual. They are structural---embedded in how research artifacts are designed, how knowledge is represented, and how incentives shape research practice.

Building on this observation, we diagnose four interrelated structural breakdowns that constrain cumulative knowledge building:
\begin{enumerate}
\item \textbf{Papers as isolated knowledge units:} Claims and evidence are embedded in prose, making direct comparison and synthesis difficult as the literature grows.
\item \textbf{Loss of context and provenance:} Motivation, assumptions, and methodological decisions fade over time, increasing the cost of building on prior work.
\item \textbf{Claims evolve without tracking:} Refinements, contradictions, and relationships between claims are documented in prose but not systematically tracked or resolved.
\item \textbf{Incentive structures favor novelty over accumulation:} Publication systems reward new contributions while undervaluing replication, consolidation, and infrastructure work.
\end{enumerate}

These breakdowns are not independent---they reinforce one another, creating a system in which knowledge accumulates slowly and inefficiently despite the community's best efforts.

In response, we articulate four principles for designing research artifacts that better support cumulative progress:
\begin{enumerate}
\item \textbf{Structured and interpretable:} Artifacts should make claims, evidence, and context explicit and directly accessible.
\item \textbf{Inspectable and provenance-aware:} Artifacts should preserve the full provenance of results and make it transparent.
\item \textbf{Long-lived and reusable:} Artifacts should support evolution and reuse beyond the moment of publication.
\item \textbf{Governed with human oversight:} Artifacts and infrastructures should be managed by community processes that ensure quality, integrity, and alignment with collective goals.
\end{enumerate}

These principles are intentionally technology-agnostic. We do not prescribe specific tools, platforms, or workflows. Instead, we describe properties that any system aiming to support knowledge accumulation should strive for, leaving room for diverse implementations across domains and research communities.

\subsection{Contributions}

This paper makes the following contributions:

\begin{itemize}
\item A community-informed diagnosis of structural barriers to knowledge accumulation in software engineering, grounded in evidence from the ICSE 2026 FOSE pre-survey.
\item An articulation of four interrelated breakdowns that explain why cumulative progress remains limited despite high levels of expertise and productivity.
\item Four principles for future research artifacts that address these breakdowns, presented as technology-agnostic guidelines rather than prescriptive solutions.
\item A discussion of implications for research practice, publication norms, community infrastructure, and the role of FOSE as an experimentation venue for alternative artifact types.
\end{itemize}

\subsection{Scope and Positioning}

This is not a tooling paper. We do not present a new system, platform, or methodology. Nor do we claim to have solved the challenges we diagnose. Instead, we offer a reflective, agenda-setting perspective appropriate for the FOSE track: a synthesis of community observations, a diagnosis of structural limitations, and a set of principles to guide future work. Our goal is to articulate why cumulative knowledge progress remains constrained and what properties future research artifacts might need to address these constraints.

We position this work as a call to rethink the fundamental design of research artifacts---not because current practices are wrong, but because they were optimized for different goals (dissemination, novelty, individual credit) and may be insufficient for the cumulative, long-term knowledge building that scientific progress requires.

\subsection{Paper Organization}

The remainder of this paper is organized as follows. Section~\ref{sec:background} situates our work within related perspectives on knowledge accumulation, reproducibility, and research infrastructure. Section~\ref{sec:survey} presents findings from the ICSE 2026 FOSE pre-survey, establishing the community profile that motivates our structural diagnosis. Section~\ref{sec:breakdown} diagnoses four interrelated breakdowns that limit cumulative progress. Section~\ref{sec:principles} articulates four principles for future research artifacts. Section~\ref{sec:implications} discusses implications for research practice, publication, and infrastructure. Section~\ref{sec:limitations} acknowledges limitations and open questions. Section~\ref{sec:conclusion} concludes with reflections on the path forward.
