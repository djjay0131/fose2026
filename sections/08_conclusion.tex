\section{Conclusion}
\label{sec:conclusion}

Software engineering research is productive and globally distributed, yet knowledge accumulation lags behind knowledge production. Leveraging insights from survey responses from the software engineering research community, this paper argues that the barriers are structural: papers function as isolated documents, provenance is lost, claims evolve without tracking, and incentives favor novelty over consolidation. Addressing these barriers requires rethinking research artifacts themselves. We have proposed four principles---\allowbreak structured and interpretable, inspectable and provenance-aware, long-lived and reusable, and governed with human oversight---to guide future work.

These principles are aspirational and technology-agnostic. Realizing them requires changes in practice, publication norms, infrastructure, and incentives. The challenges are significant, but the alternative---continuing with practices optimized for dissemination rather than accumulation---will perpetuate fragmentation as the field grows.

The Future of Software Engineering track provides space for this conversation. Next steps include experimenting with alternative artifact designs, developing infrastructure aligned with these principles, and revising recognition systems to value consolidation alongside novelty. Most importantly, the community must engage in dialogue about what cumulative progress requires and what trade-offs are acceptable.

The future of software engineering depends not only on what we discover but on how we organize and preserve what we know. Building infrastructures that support cumulative knowledge building---making it rewarded, not just possible---is essential to ensuring the next generation inherits not just a literature to read but a knowledge base to build upon.
