\section{Conclusion}
\label{sec:conclusion}

Software engineering research is thriving by many measures. The field is productive, the community is experienced, and participation spans the globe. Yet this success in producing research outputs has not translated fully into cumulative knowledge progress. Claims remain fragmented across disconnected papers. Context and provenance are lost as knowledge moves through the publication pipeline. Synthesis and reuse remain effortful and error-prone. The mechanisms we have for accumulating knowledge have not kept pace with the mechanisms we have for producing it.

This paper has argued that the barriers to cumulative progress are structural, not individual. They arise from how research artifacts are designed, how knowledge is represented, and how incentives shape practice. If papers function as isolated documents, if provenance is not preserved, if claims evolve without tracking, and if novelty is rewarded over consolidation, then accumulation will remain limited no matter how skilled or motivated the community.

We have diagnosed four interrelated breakdowns and proposed four principles to address them. Structured and interpretable artifacts make claims and evidence directly accessible. Inspectable and provenance-aware artifacts preserve the reasoning behind results. Long-lived and reusable artifacts support evolution beyond publication. Governed artifacts align individual incentives with collective knowledge-building goals.

These principles are aspirational. Realizing them would require changes in research practice, publication norms, community infrastructure, and incentive systems. We do not underestimate the challenges. Building shared infrastructures, establishing standards, reforming review processes, and changing institutional priorities are difficult, time-consuming, and require coordination across many actors.

But the alternative---continuing with practices optimized for dissemination and novelty rather than accumulation and reuse---will perpetuate the fragmentation and inefficiency we currently experience. As the field continues to grow, these challenges will only intensify. What is manageable today may become unmanageable tomorrow.

The Future of Software Engineering track is an ideal venue for this conversation. FOSE papers are not expected to provide complete solutions or empirical validation. They are expected to be reflective, forward-looking, and agenda-setting. This paper has aimed to meet that expectation: to articulate why cumulative progress remains constrained, to propose principles for addressing these constraints, and to discuss implications for how the community might move forward.

We see several potential next steps. First, the community could experiment with alternative artifact types and practices, using FOSE and other venues as spaces for prototyping and evaluation. Second, infrastructure initiatives---knowledge repositories, benchmark collections, collaborative platforms---could be developed with explicit attention to the principles we have articulated. Third, conferences, journals, and institutions could revisit policies and criteria to recognize contributions that consolidate, replicate, and support cumulative building alongside those that introduce novelty.

Most importantly, the community could engage in dialogue about what cumulative progress means, what it requires, and what trade-offs are acceptable. This paper offers one perspective, grounded in survey data and informed by related work. But it is not the final word. Other perspectives, critiques, and refinements are essential.

The software engineering research community has demonstrated its capacity for growth, adaptation, and innovation. The field has evolved in response to changing technologies, expanding application domains, and new research methods. Rethinking how we design research artifacts and build knowledge infrastructures is the next evolution---one that could make the difference between sustained productivity and sustained progress.

The question is not whether cumulative knowledge building is possible. It is whether the community will prioritize it, invest in it, and organize around it. The principles we have articulated provide a starting point. The path from principles to practice will require effort, experimentation, and commitment. But the potential reward---a research community that not only produces knowledge but also accumulates, integrates, and reuses it effectively---is worth pursuing.

The future of software engineering depends not only on what we discover but on how we organize and preserve what we know. Let us build infrastructures and practices that make cumulative progress not just possible but rewarded, supported, and celebrated. Let us ensure that the next generation of researchers inherits not just a literature to read but a knowledge base to build upon.
