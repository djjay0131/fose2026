\section{Implications for the Future of Software Engineering}
\label{sec:implications}

Realizing these principles would require changes in how research is conducted, reviewed, published, and rewarded.

\subsection{Research Practice and Publication}

Research practice would shift toward documenting process alongside outcomes: recording methodological decisions, their rationales, and impacts using computational notebooks, version control, and workflow platforms. Researchers would supplement narrative papers with structured representations---semantic annotations or knowledge graph entries---making contributions directly accessible for synthesis. Designing for reuse would become standard: datasets documented with schemas, code modular and documented, protocols reproducible. Structured artifacts would enable new collaboration forms, with researchers contributing incrementally to shared knowledge bases rather than isolated papers.

Publication and review would need to value consolidation alongside novelty. Papers that resolve contradictions, curate benchmarks, or provide infrastructure should be recognized. Conferences could establish expectations for structured artifacts as first-class contributions, not optional supplements. If artifacts evolve post-publication, papers become snapshots while living artifacts accumulate evidence and track refinements, requiring new norms for citing evolving work.

\subsection{Community Infrastructure}

The community would need platforms for storing, querying, and updating structured artifacts---knowledge graphs, benchmark repositories, collaborative platforms---maintained through community governance. Shared standards (ontologies, schemas, reporting protocols) would enable reuse across studies. Infrastructure contributions, replications, and refinements would require recognition through alternative metrics, tenure criteria changes, or contribution-tracking platforms. Governance structures would manage stewardship: maintaining repositories, resolving disputes, updating standards, addressing ethical concerns. Building such infrastructure requires sustained investment and coordination, but without it, the principles remain aspirational.

\subsection{FOSE as a Venue for Experimentation}

The Future of Software Engineering track is positioned to support experimentation with alternative artifact types. FOSE could encourage submissions experimenting with structured artifacts or living documents, judged on potential to demonstrate new knowledge-organizing approaches rather than solely novelty. FOSE sessions could facilitate dialogue about infrastructure needs and governance models, and track experiments over time to inform broader community decisions about adopting new norms.
