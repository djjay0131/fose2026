\section{Implications for the Future of Software Engineering}
\label{sec:implications}

The principles articulated in the previous section are aspirational. Realizing them would require changes not only in how research artifacts are designed, but also in how research is conducted, reviewed, published, and rewarded. In this section, we discuss implications for research practice, publication norms, community infrastructure, and the role of FOSE as a venue for experimentation.

\subsection{Implications for Research Practice}

If research artifacts were structured, inspectable, provenance-aware, and long-lived, how would research practice change?

\textbf{Planning and documentation.} Researchers would need to document not only final results but also the process by which those results were obtained. This includes recording methodological decisions, their rationales, and their impacts. Tools like computational notebooks, version control systems, and workflow management platforms could support this documentation, but the cultural shift---valuing process alongside outcomes---is equally important.

\textbf{Structuring contributions.} Instead of presenting results solely in prose, researchers would also produce structured representations of claims, evidence, and context. This does not mean abandoning narrative---papers would still tell stories, explain motivations, and argue for significance. But they would be supplemented by machine-readable metadata, semantic annotations, or knowledge graph entries that make key contributions directly accessible for synthesis and reuse.

\textbf{Designing for reuse.} Researchers would consider reusability at the outset, not as an afterthought. Datasets would be documented with schemas and provenance. Code would be modular and well-documented. Evaluation protocols would be reproducible. The goal would be to produce artifacts that others can directly build on, reducing the need for reimplementation and lowering the barrier to cumulative progress.

\textbf{Collaborating across studies.} Structured artifacts enable new forms of collaboration. Researchers could contribute incrementally to shared knowledge bases, updating claims as new evidence emerges, proposing refinements to existing techniques, or resolving contradictions by comparing structured evidence. This would shift the unit of contribution from the paper to the structured artifact, enabling continuous, distributed knowledge building.

These changes would require time and effort. But if the infrastructure and incentives were in place, they could become standard practice, much as sharing code and data have become increasingly expected in many research communities.

\subsection{Implications for Publication and Review}

If the principles were adopted, what would change in how research is published and reviewed?

\textbf{Evaluating contributions beyond novelty.} Review processes would need to recognize contributions that consolidate, synthesize, or replicate prior work. A paper that resolves contradictions in the literature, curates a benchmark dataset, or provides infrastructure for cumulative building should be valued alongside papers that propose new techniques. This requires updating review criteria and training reviewers to assess different types of contributions.

\textbf{Artifact expectations.} Conferences and journals could establish expectations that submissions include not only papers but also structured artifacts: semantic annotations, provenance documentation, reusable components. Artifact evaluation tracks have made progress in this direction, but the principles suggest going further---treating structured artifacts as first-class contributions, not optional supplements.

\textbf{Living publications.} If artifacts evolve beyond publication, then papers could be viewed as snapshots of ongoing work rather than final records. A paper might introduce a technique and present initial results, while a living artifact continues to accumulate evidence, track refinements, and document extensions. This would require rethinking version control for scholarly communication and establishing norms for how to cite evolving artifacts.

\textbf{Community review and curation.} Governance mechanisms imply that some contributions would be reviewed not only at submission time but also post-publication, as they are integrated into shared knowledge bases. Community curation processes---similar to those used in some open-access repositories or collaborative platforms---could ensure that contributions meet quality standards, resolve conflicts, and maintain consistency over time.

These changes would be significant, but they are not without precedent. Fields like machine learning have experimented with leaderboards, benchmark repositories, and living evaluation frameworks. The principles suggest extending these experiments more broadly and embedding them in the publication process itself.

\subsection{Implications for Community Infrastructure}

If the principles were realized, what infrastructure would be needed?

\textbf{Shared knowledge repositories.} The community would need platforms for storing, querying, and updating structured research artifacts. These could take many forms: knowledge graphs representing claims and evidence; benchmark repositories hosting datasets and evaluation scripts; collaborative platforms where researchers propose and refine contributions. The key is that these repositories would be long-lived, maintained, and governed by community processes.

\textbf{Standards and interoperability.} For artifacts to be reusable across studies, the community would need shared standards: ontologies for representing claims and techniques, schemas for documenting datasets, protocols for reporting results. Establishing these standards requires coordination, but the benefits---reduced friction in synthesis and reuse---could be substantial.

\textbf{Credit and attribution systems.} Infrastructure contributions, replication efforts, and incremental refinements would need to be recognized and credited. This could involve alternative metrics (e.g., how often a dataset is reused, how many claims reference a benchmark), changes in tenure and promotion criteria, or platforms that track contributions beyond traditional papers. Aligning individual incentives with collective knowledge-building goals is essential for sustainability.

\textbf{Governance and stewardship.} Shared infrastructures require stewardship: maintaining repositories, resolving disputes, updating standards, addressing ethical concerns. The community would need governance structures---committees, boards, or distributed processes---to manage these responsibilities. Transparency, inclusivity, and accountability would be critical to ensuring that infrastructures serve the collective good.

Building and maintaining such infrastructure is not trivial. It requires sustained investment, coordination across institutions and subfields, and a willingness to prioritize collective needs alongside individual research agendas. But without infrastructure, the principles remain aspirational.

\subsection{FOSE as a Venue for Experimentation}

The Future of Software Engineering track is uniquely positioned to support experimentation with alternative artifact types and practices. FOSE papers do not need to conform to the same evaluation criteria as empirical studies or tool papers. They can be reflective, speculative, and forward-looking. This creates space for contributions that would not fit traditional tracks but that could inform the evolution of the field.

\textbf{Prototyping alternative formats.} FOSE could encourage submissions that experiment with structured artifacts, living documents, or collaborative knowledge-building platforms. These contributions would not be judged solely on novelty or empirical rigor but on their potential to demonstrate new ways of organizing and sharing knowledge.

\textbf{Community dialogues.} FOSE sessions could serve as forums for discussing infrastructure needs, governance models, and incentive alignment. The track could facilitate conversations that bridge research practice, publication systems, and community infrastructure---conversations that are difficult to have within the constraints of traditional paper sessions.

\textbf{Long-term tracking.} FOSE could track experiments over time, documenting what works, what fails, and what lessons emerge. This would provide evidence for evaluating alternative practices and informing broader community decisions about adopting new norms or infrastructures.

By embracing experimentation, FOSE can help the software engineering community move from principles to practice, testing ideas, refining approaches, and building momentum for structural change.

\subsection{Challenges and Trade-offs}

Realizing these implications would not be without challenges. Structured artifacts require effort to produce and maintain. Standards risk becoming rigid or exclusionary. Governance processes can be contentious. Infrastructure requires resources that may not be available or equitably distributed.

Moreover, there are trade-offs. Emphasizing structure and reusability could discourage exploratory, early-stage work that does not yet fit established frameworks. Focusing on consolidation could slow the introduction of genuinely novel ideas. Balancing these tensions---supporting both exploration and consolidation, novelty and cumulation, individual creativity and collective building---is an ongoing challenge.

The goal is not to replace current practices wholesale but to expand the range of valued contributions and supported practices. Papers will remain important for storytelling and argumentation. But they could be complemented by structured artifacts, living repositories, and community infrastructures that make cumulative progress more feasible and more rewarded.
