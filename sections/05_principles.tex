\section{Rethinking Research Artifacts for Cumulative Progress}
\label{sec:principles}

The structural barriers diagnosed in the previous section are not inevitable. They arise from design choices embedded in how research artifacts are currently produced and shared. If papers-as-documents create fragmentation, and if current incentive structures discourage consolidation, then addressing these challenges requires reconsidering the fundamental properties that research artifacts should have. In this section, we articulate four principles to guide the design of artifacts that support cumulative, interpretable, and reusable knowledge building.

These principles are intentionally abstract. We do not prescribe specific technologies, platforms, or workflows. Instead, we describe properties that any system, tool, or practice aiming to support knowledge accumulation should strive for. Concrete implementations will vary across domains and communities, but the underlying principles provide a shared foundation for evaluating and improving research infrastructure.

\subsection{Principle 1: Structured and Interpretable}

\textbf{Principle:} Research artifacts should make claims, evidence, and context explicit and directly accessible, not only embedded in prose.

The first breakdown---papers as isolated units---stems from the fact that claims and evidence are woven into narrative text. While prose is valuable for explaining motivation and argumentation, it is not optimal for supporting cumulative synthesis. A reader wishing to compare claims across papers must extract and interpret statements from natural language, a process that does not scale as the literature grows.

\textbf{What this principle requires.} Artifacts should represent claims, evidence, and context as structured, first-class entities. A claim should be identifiable: ``Technique X improves metric Y on dataset Z by $\delta$ under conditions C.'' The evidence supporting that claim---experimental configuration, data, results, analysis---should be explicitly linked. The context---assumptions, limitations, methodological decisions---should be preserved alongside the claim, not buried in prose.

Structured representations enable direct comparison, aggregation, and reasoning over results. If two papers make conflicting claims about the same technique, the conflict becomes visible and resolvable by examining the structured evidence and context. If a dataset is updated or a metric refined, claims referencing that dataset or metric can be systematically identified and reevaluated.

\textbf{Instantiation examples.} Structured representations could take many forms:
\begin{itemize}
\item \textit{Semantic annotations:} Papers supplemented with machine-readable metadata describing claims, experimental setups, and results.
\item \textit{Knowledge graph representations:} Claims, techniques, datasets, and metrics represented as entities with typed relationships, enabling queries like ``Which papers claim improvements on dataset D using technique family T?''
\item \textit{Computational notebooks with assertions:} Executable analyses in which key claims are formalized as assertions that can be tested, versioned, and reused.
\end{itemize}

These are illustrative, not prescriptive. The principle is that structure should complement prose, making knowledge directly accessible for synthesis while preserving the explanatory power of narrative.

\subsection{Principle 2: Inspectable and Provenance-Aware}

\textbf{Principle:} Research artifacts should preserve the full provenance of claims---from raw data through methodological decisions to final results---and make this provenance inspectable.

The second breakdown---loss of context and provenance---occurs because the reasoning behind decisions fades as knowledge moves through the publication pipeline. A paper reports final results but often omits the path taken to reach them: why this dataset, why this baseline, why this evaluation protocol. When later researchers attempt to build on the work, they must reconstruct this reasoning, often discovering that subtle choices had significant consequences.

\textbf{What this principle requires.} Artifacts should document not only what was found, but how and why. Every claim should trace back to its sources: the data, the code, the configuration, the assumptions. Methodological decisions should be explicitly recorded and justified: ``We used dataset D because it contains feature F relevant to hypothesis H.'' When results are updated---due to new data, corrected analyses, or refined methods---the provenance chain should track these changes, preserving the history of how understanding evolved.

Provenance enables trust and reproducibility. A researcher can inspect the lineage of a claim, verify that it rests on sound evidence, and understand the conditions under which it holds. When claims conflict, provenance helps diagnose the source of divergence: different data, different preprocessing, different evaluation criteria.

\textbf{Instantiation examples.} Provenance tracking could be realized through:
\begin{itemize}
\item \textit{Versioned computational artifacts:} Code, data, and configurations stored in version control systems with clear lineage from raw inputs to published results.
\item \textit{Provenance graphs:} Explicit representations of how results were derived, linking claims to experiments, experiments to data, and data to sources.
\item \textit{Decision logs:} Structured records of methodological choices, their rationales, and their impacts, made part of the permanent artifact.
\end{itemize}

Again, these are examples. The principle is that artifacts should be inspectable---transparent about their origins and decisions---so that future work can be built on solid, well-understood foundations.

\subsection{Principle 3: Long-Lived and Reusable}

\textbf{Principle:} Research artifacts should support evolution and reuse, not remain static at the moment of publication.

The third breakdown---claims evolve without tracking---and aspects of the first breakdown arise because papers are snapshots frozen in time. Once published, a paper does not update when new evidence emerges, when datasets are revised, or when methods are improved. Follow-on work cites the paper and describes differences in prose, but the original claim remains unchanged in the literature.

\textbf{What this principle requires.} Artifacts should be living substrates that can be updated, extended, and reused. When a dataset is corrected, claims depending on that dataset should be re-evaluable, not obsolete. When a technique is refined, prior results should be comparable to new results using updated methods. When claims are qualified or contradicted, these relationships should be reflected in the artifact, not scattered across disconnected papers.

This does not mean papers should be continuously rewritten. Rather, the underlying structured representations---claims, evidence, configurations---should be decoupled from narrative documents. Narratives can remain stable as historical records, while the structured substrates evolve as understanding progresses.

\textbf{Instantiation examples.} Long-lived artifacts could include:
\begin{itemize}
\item \textit{Living knowledge bases:} Repositories where claims, datasets, and techniques are maintained, versioned, and updated as new evidence accumulates.
\item \textit{Executable benchmarks:} Evaluation frameworks that can be rerun with updated data or baselines, allowing claims to be retested and results to be continuously refined.
\item \textit{Modular artifact ecosystems:} Components (datasets, models, evaluation scripts) designed for composition and reuse, enabling new studies to build directly on prior artifacts rather than reimplementing from scratch.
\end{itemize}

The principle is that knowledge should outlive individual papers, accumulating in shared substrates that reduce redundancy and enable direct building.

\subsection{Principle 4: Governed with Human Oversight}

\textbf{Principle:} Research artifacts and infrastructures should be governed by community processes that ensure quality, integrity, and ethical responsibility.

The fourth breakdown---incentive structures favor novelty---reflects the tension between individual incentives and collective needs. Even if we had perfect technical infrastructure, cumulative progress requires coordination: standards for quality, mechanisms for resolving conflicts, and recognition for contributions that consolidate rather than merely add.

\textbf{What this principle requires.} Artifacts cannot govern themselves. Structured representations, provenance tracking, and living substrates are valuable only if the community trusts them, maintains them, and uses them responsibly. This requires human oversight: peer review for knowledge contributions, curation of shared resources, community processes for resolving disputes or establishing standards, and credit systems that value infrastructure work alongside novel research.

Governance also addresses ethical concerns. Structured knowledge artifacts raise questions about ownership, attribution, bias, and access. Who controls a shared knowledge base? How is credit assigned when multiple researchers contribute incrementally? How do we ensure that knowledge infrastructures do not perpetuate biases present in underlying data? These are social and ethical questions, not purely technical ones, and they require community deliberation and ongoing stewardship.

\textbf{Instantiation examples.} Governance mechanisms could include:
\begin{itemize}
\item \textit{Community curation processes:} Peer-reviewed contributions to shared knowledge bases, with clear criteria for inclusion, updating, and deprecation.
\item \textit{Credit and attribution systems:} Mechanisms that recognize contributions to infrastructure, replication, and consolidation, not only novel results.
\item \textit{Ethical oversight boards:} Community structures responsible for addressing bias, fairness, and access issues in shared knowledge infrastructures.
\item \textit{Transparent governance models:} Decision-making processes for managing shared resources, resolving conflicts, and setting standards, with broad community input.
\end{itemize}

The principle is that cumulative progress is a collective endeavor. Technical solutions alone are insufficient; they must be embedded in social practices that align individual incentives with collective goals and ensure responsible stewardship of shared knowledge.

\subsection{Mapping Principles to Breakdowns}

These four principles are not independent---they address the interconnected breakdowns diagnosed earlier:

\begin{itemize}
\item \textbf{Structured and interpretable} artifacts address the isolation of knowledge by making claims and evidence directly comparable and composable.
\item \textbf{Inspectable and provenance-aware} artifacts address the loss of context by preserving the reasoning and decisions behind results.
\item \textbf{Long-lived and reusable} artifacts address the static nature of claims by enabling evolution and incremental refinement over time.
\item \textbf{Governed with human oversight} addresses incentive misalignment by establishing community processes that value consolidation and ensure responsible stewardship.
\end{itemize}

Together, they describe a vision for research artifacts that support cumulative knowledge building: artifacts that are not isolated documents but interconnected, evolving substrates governed by community practices that align individual contributions with collective progress.

\subsection{From Principles to Practice}

These principles are aspirational. Realizing them fully would require significant changes in how research is conducted, reviewed, published, and rewarded. We do not claim that these changes are easy or that any single intervention will suffice. But the principles provide a framework for evaluating incremental steps: Does a proposed tool, platform, or practice make artifacts more structured, more inspectable, more reusable, or better governed? If so, it moves the community toward more cumulative progress.

Importantly, these principles are not tied to specific technologies. Knowledge graphs, computational notebooks, version control systems, and community repositories are all potential instantiations---but the principles themselves are technology-agnostic. They describe \textit{properties} that artifacts should have, leaving room for diverse implementations across different domains and research communities.

The challenge is to move from principles to practice: to design systems, establish norms, and reform incentives in ways that make cumulative knowledge building not only possible but rewarded. We turn to the broader implications of this shift in the next section.
