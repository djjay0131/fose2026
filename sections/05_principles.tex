\section{Rethinking Research Artifacts for Cumulative Progress}
\label{sec:principles}

The structural barriers arise from design choices in how artifacts \todo{define artifacts (and if/how they are related to knowledge accumulation)} are produced and shared. Addressing them requires reconsidering fundamental artifact properties. We articulate four technology-agnostic principles describing properties that systems, tools, or practices aiming to support knowledge accumulation should strive for. Implementations will vary across domains, but these principles provide a shared foundation for evaluating research infrastructure.

\subsection{Principle 1: Structured and Interpretable}

\textbf{Principle:} Research artifacts should make claims, evidence, and context explicit and directly accessible, not only embedded in prose.

The first breakdown---papers as isolated units---stems from the fact that claims and evidence are woven into narrative text. While prose is valuable for explaining motivation and argumentation, it is not optimal for supporting cumulative synthesis. A reader wishing to compare claims across papers must extract and interpret statements from natural language, a process that does not scale as the literature grows.

Artifacts should represent claims, evidence, and context as structured, first-class entities. A claim should be identifiable: ``Technique X improves metric Y on dataset Z by $\delta$ under conditions C.'' Evidence and context should be explicitly linked, not buried in prose. Structured representations enable direct comparison and reasoning: conflicting claims become visible and resolvable by examining structured evidence, and claims referencing updated datasets can be systematically reevaluated. Instantiations could include semantic annotations (\ie machine-readable metadata describing claims and results) or knowledge graphs~\cite{orkg,wang2020,stocker2024orkg,wang2023kg} (\ie entities with typed relationships enabling queries like ``Which papers claim improvements on dataset D?''). Structure should complement prose, making knowledge directly accessible while preserving narrative's explanatory power.

\subsection{Principle 2: Inspectable and Provenance-Aware}

\textbf{Principle:} Research artifacts should preserve the full provenance of claims---from raw data through methodological decisions to final results---and make this provenance inspectable \todo{what does inspectable mean?}.

The second breakdown---loss of context and provenance---occurs because reasoning behind decisions fades as knowledge moves through publication. Papers report final results but often omit the path taken: why this dataset, baseline, or protocol. Later researchers must reconstruct this reasoning, often discovering subtle choices had significant consequences.

Artifacts should document not only what was found but how and why. Every claim should trace to its sources: data, code, configuration, assumptions. Methodological decisions should be explicitly justified, and when results are updated, the provenance chain should track changes~\cite{gonzalez-barahona2012}. Provenance enables trust: researchers can inspect a claim's lineage, verify evidence, and understand conditions. When claims conflict, provenance diagnoses divergence sources. Instantiations could include versioned computational artifacts (\ie code and data in version control with clear lineage) or provenance graphs (\ie explicit representations linking claims to experiments to data). Artifacts should be transparent about origins and decisions, enabling future work to build on solid foundations.

\subsection{Principle 3: Long-Lived and Reusable}

\textbf{Principle:} Research artifacts should support evolution and reuse, not remain static at the moment of publication.

Papers are snapshots frozen in time. Once published, they rarely update when new evidence emerges, datasets are revised, or methods improve. Follow-on work cites and describes differences in prose, but original claims remain unchanged. Further, research artifacts are often difficult or impossible to reuse. For example, prior efforts to replicate research systems published in ICSE and FSE tool demonstration tracks succeeded in running only about half of the tools (76/131), despite investing thousands of hours in installing, configurating, and executing the tools~\cite{murphyhill-reviving}.

Artifacts should be living substrates that can be updated, extended, and reused. When datasets are corrected, dependent claims should be re-evaluable. When techniques are refined, prior results should be comparable to new results. When claims are qualified, these relationships should be reflected in the artifact. This does not mean rewriting papers---rather, underlying structured representations should decouple from narrative documents. Narratives remain stable as historical records while structured substrates evolve. Instantiations could include living knowledge bases (\ie repositories where claims and datasets are versioned and updated) or executable benchmarks (\ie evaluation frameworks rerun with updated data). Knowledge should outlive individual papers, accumulating in shared substrates that reduce redundancy and enable direct building.

\subsection{Principle 4: Governed with Human Oversight}

\textbf{Principle:} Research artifacts and infrastructures should be governed by community processes that ensure quality, integrity, and ethical responsibility.

Even with perfect technical infrastructure, cumulative progress requires coordination: quality standards, conflict resolution mechanisms, and recognition for consolidation work. Artifacts cannot govern themselves---structured representations and provenance tracking are valuable only if the community trusts, maintains, and uses them responsibly. This requires human oversight: peer review for knowledge contributions, curation, dispute resolution processes, and credit systems valuing infrastructure work alongside novel research. For instance, the human-artifact models suggests human expertise is necessary to understand artifacts within the larger artifact ecosystem and the evolving technological landscapes, and motivate future designs~\cite{b2011human}.

Governance also addresses ethical concerns about ownership, attribution, bias, and access. Who controls shared knowledge bases? How is credit assigned for incremental contributions? How do we prevent infrastructures from perpetuating biases? These social and ethical questions require community deliberation and ongoing stewardship. Instantiations could include community curation processes (\ie peer-reviewed contributions with clear criteria) or credit systems recognizing infrastructure, replication, and consolidation work. Cumulative progress is a collective endeavor---technical solutions must be embedded in social practices aligning individual incentives with collective goals.

% \subsection{From Principles to Practice}

% These principles are aspirational, requiring significant changes in how research is conducted, reviewed, and rewarded. They provide a framework for evaluating incremental steps: Does a proposed tool or practice make artifacts more structured, inspectable, reusable, or better governed? The principles are technology-agnostic---\allowbreak knowledge graphs, computational notebooks, and repositories are potential instantiations, but the principles describe \textit{properties} that artifacts should have, leaving room for diverse implementations. The challenge is designing systems and reforming incentives to make cumulative knowledge building not only possible but rewarded.
