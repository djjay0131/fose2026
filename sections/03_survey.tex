\section{Community Signals from the ICSE 2026 FOSE Pre-Survey}
\label{sec:survey}

To ground our discussion in community perspectives, we draw on responses from the ICSE 2026 Future of Software Engineering track pre-survey. This survey was designed to understand how members of the software engineering research community perceive current challenges and opportunities in the field. We received 280 completed responses from researchers representing diverse roles, experience levels, and geographic regions.

Our purpose in examining this survey is not to conduct a comprehensive empirical study, but rather to establish context and legitimacy for the structural challenges we diagnose in this paper. The survey data reveals a community profile that is critical to our central argument: if knowledge accumulation remains limited despite significant expertise and participation, then the barriers must be structural rather than individual.

\subsection{Community Profile}

The survey respondents represent a mature, active, and globally distributed research community. Table~\ref{tab:survey-demographics} summarizes key demographic characteristics.

\begin{table}[t]
\caption{FOSE 2026 pre-survey respondent demographics (n=280)}
\label{tab:survey-demographics}
\begin{tabular}{lrr}
\toprule
\textbf{Characteristic} & \textbf{Count} & \textbf{Percentage} \\
\midrule
\multicolumn{3}{l}{\textit{Years in SE Research Community}} \\
\quad 0--1 years & 12 & 4.3\% \\
\quad 2--3 years & 26 & 9.3\% \\
\quad 4--5 years & 29 & 10.4\% \\
\quad 6--10 years & 50 & 17.9\% \\
\quad 11--20 years & 82 & 29.3\% \\
\quad 21+ years & 78 & 27.9\% \\
\midrule
\multicolumn{3}{l}{\textit{Papers Authored (Past 3 Years)}} \\
\quad 0 papers & 8 & 2.9\% \\
\quad 1--3 papers & 54 & 19.3\% \\
\quad 4--10 papers & 91 & 32.5\% \\
\quad 11--20 papers & 73 & 26.1\% \\
\quad 21--40 papers & 32 & 11.4\% \\
\quad 41+ papers & 19 & 6.8\% \\
\midrule
\multicolumn{3}{l}{\textit{Geographic Distribution}} \\
\quad Europe & 139 & 49.6\% \\
\quad North America & 70 & 25.0\% \\
\quad Asia & 41 & 14.6\% \\
\quad South America & 14 & 5.0\% \\
\quad Oceania & 11 & 3.9\% \\
\quad Middle East & 2 & 0.7\% \\
\bottomrule
\end{tabular}
\end{table}

\textbf{Experience.} The majority of respondents (57.1\%) have been involved in software engineering research for over a decade, with 29.3\% reporting 11--20 years of experience and 27.9\% reporting 21 or more years. This is not a community of newcomers struggling to establish research programs, but rather one populated by seasoned researchers who have observed and participated in the field's evolution over substantial periods.

\textbf{Productivity.} The community is actively engaged in producing research outputs. Nearly half of respondents (44.3\%) reported authoring 11 or more paper submissions in the past three years alone---an average of at least 3--4 papers per year. An additional 32.5\% reported 4--10 submissions. Only 2.9\% reported zero submissions. This high level of productivity suggests a community that is not constrained by lack of effort or output capacity.

\textbf{Global Reach.} Respondents represent six geographic regions spanning Europe, North and South America, Asia, Oceania, and the Middle East. While Europe is most heavily represented (49.6\%), North American (25.0\%) and Asian (14.6\%) researchers form substantial communities. This global distribution indicates that the challenges we discuss are not artifacts of a single research culture or institutional context, but rather reflect patterns that transcend geographic boundaries.

\subsection{Implications for Knowledge Accumulation}

The demographic profile of survey respondents carries a critical implication for our argument. The software engineering research community is not lacking in \textit{expertise}---over half the respondents have more than a decade of experience. It is not lacking in \textit{participation}---nearly half produce at least 11 paper submissions every three years. And it is not limited to a narrow geographic or cultural context---six regions are represented, with substantial participation from multiple continents.

If knowledge accumulation remains fragmented, if claims and evidence remain weakly linked, and if rediscovery costs remain high despite this level of experience, productivity, and global engagement, then the barriers to cumulative progress must be \textit{structural}. They cannot be attributed to insufficient expertise, inadequate effort, or limited participation. Instead, they must arise from the ways in which research artifacts are produced, represented, and connected to one another over time.

\subsection{Recognized Challenges}

Beyond demographics, the survey responses reveal that the community itself recognizes systemic challenges. When asked what aspects of the software engineering research community do not work well, respondents frequently pointed to issues related to knowledge synthesis and cumulative building:

\begin{quote}
\textit{``Reviewing process: quality of reviews is not always good---too many papers get submitted (and resubmitted).''}
\end{quote}

\begin{quote}
\textit{``We're not inclusive or sustainable enough insofar as the remote participation is lousy. I'd do away with all of our program committees and just do journal first for everything. I'd like to see people have more flexibility in which and whether to attend conferences and less waste and resubmission and re-review of papers.''}
\end{quote}

\begin{quote}
\textit{``Adopting new technologies. Let's say NLP community already adopted LLM based annotation with human supervision... yet in SE papers 30\% of the justification goes to why this technology works well. It's like doing two research in a single paper.''}
\end{quote}

These comments hint at deeper structural issues: cycles of resubmission that fragment results across venues and versions, difficulties in building on prior work efficiently, and challenges in tracking how ideas evolve as they move through the publication pipeline. Respondents also expressed concerns about review quality, the stress of managing high submission volumes, and the difficulty of preparing mentees for a system that rewards novelty over consolidation.

\subsection{Transition to Structural Diagnosis}

The survey data establishes that the software engineering research community possesses significant expertise, maintains high productivity, and operates across global contexts. Yet respondents themselves recognize persistent challenges in how knowledge is synthesized, reviewed, and built upon. These observations set the stage for a more detailed examination of the structural barriers that limit cumulative knowledge accumulation---the subject of the next section.
